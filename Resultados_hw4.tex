\documentclass[a4paper,11pt]{article}

\usepackage[latin1]{inputenc}
\usepackage{graphicx}
\usepackage{color}

\begin{document}

\begin{center}
{\Large Tarea 4 m\'etodos computacionales} \\
{\textbf Liliana Quintero -201718286}
\end{center}


\noindent
\section{Gr\'aficas proyectil}

\subsection{Angulo 45 grados}

\begin{center}
\includegraphics[width=8cm]{Plot_ODE_1.pdf} 
Gr\'afica para el proyectil lanzado con un angulo de 45 grados, la distancia m\'axima obtenida es de x=4.26, la altura m\'axima es de y=3.42. Se puede ver como la gravedad y la resistencia del aire forman el tiro parab\'olico y adem\'as como debido a la fricci\'on tanto en x como en y se curva el movimiento de forma que no es una parabola 'perfecta'.
\end{center}

\subsection{Diferentes angulos}

\begin{center}
\includegraphics[width=8cm]{Plot_ODE_2.pdf}
\end{center}
Gr\'afica para el proyectil lanzado con diferentes \'angulos y se puede observar como el \'angulo para el cual se obtuvo la mayor distancia en x fue de 20 grados con el cual la distancia recorrida con el proyectil fue de x=5.23. Como despu\'ues de este \'angulo las distancias recorridas dismiyunes a medida que el \'angulo aumenta, la distancia m\'as corta presentada fue para el \'angulo de 70 grados con el cual se obtuvo una distancia de x=2.22.  Igulamente se observa como el tiro no es perfectamente parab\'olico debido a la fricci\'on presente tanto en el eje x como y.

\section{Gr\'aficas difusi\'on t\'ermica}

Las siguientes gr\'aficas de difusi\'on representan un sistema qeu costa de una piedra caliza que inicialmente se encuentra a temperatura 10 grados y se pone en contacto con una barilla perpendicular en el centro de la misma, y se observa el cambio de la temperatura atravez de la piedra a medida que pasa el tiempo. La piedra tiene un nu espec\'ifico (coeficiente __).


\subsection{Fronteras fijas}

Para las fronteras fijas tenemos que todos los 'bordes' del cuadrado se encuentran siempre a una temperatura de 10 grados.

\begin{center}
\includegraphics[width=8cm]{1.pdf}
\end{center}
Esta es la condici\'on inicial para la piedra, la barilla se encuentra a 100 grados y la piedra a 10 grados para un tiempo t=0.

\includegraphics[width=6cm]{2.pdf}
\includegraphics[width=6cm]{3.pdf}
Estas dos gr\aficas muestran el cambio de la temperatura para dos tiempos intermedios diferentes en los cuales se puede observar el aumento general de temperatura de la misma. En las gr\'aficas se puede observar como se va dando el aumento de la temperatura a lo largo de toda la piedra.

\begin{center}
\includegraphics[width=8cm]{4.pdf}
\end{center}
Esta gr\'afica muestra la condici\'on de equilibrio encontrada, nunca llegan a 100 ya que los extremos son fijos sin embargo llega un punto en el que ya no aumenta o dosminuye m\as la temperatura. Queda en un punto estable.

\begin{center}
\includegraphics[width=8cm]{Promedio1.pdf}
\end{center}
La grafica dle promedio de temperaturas a traves del tiempo muestra que la temperatura va aumentando hasta que el gradiente de cambio es muy bajo y la temperatura deja de aumentar que es cuando consideramos el equilibrio termico.

\subsection{Fronteras abiertas}
Para fronteras abiertas las part\'iculas de los extremos se comportan libremente como el resto de las part\'iculas.

\begin{center}
\includegraphics[width=8cm]{5.pdf}
\end{center}
La condici\'on inicial es la misma qeu la de fronteras fijas con la barilla a 100 y el resto a 10
\includegraphics[width=6cm]{6.pdf}
\includegraphics[width=6cm]{7.pdf}
Estas gr\'aficas muestran los puntos medios ya se puede observar que debido a que las fronteras no son fijas la placa completa va aumentando su temperatura para eventualmente llegar a la temperatura de equilibrio.

\begin{center}
\includegraphics[width=8cm]{8.pdf}
\end{center}
Esta gr\'afica muestra el cambio de temperatura al pasar 500.000 segundos, la temperatura de equilibrio se encuentra m\'as arriba cuando toda la placa llegue a 100 grados, sin embargo debido al gran n\'umero de iteraciones necesarias en el c\'odigo se us\'o este valor y no el final.

\begin{center}
\includegraphics[width=8cm]{Promedio3.pdf}
\end{center}
Se puede ver que el cambio de temperatura a traves del tiempo es mas lineal que el caso de fronteras fijas, y hay un aumento continuo, en la grafica no se alcanza a observar la temperatura de equilibrio qeu es cuando deja de crecer la recta y la pendiente se vuelve 0 esto debido a que el numero de iteraciones no se uso para alcanzar dicho equilibrio.

\subsection{Fronteras peri\'odicas}
Para las fronteras peri\'odicas es como considerar que nuestra placa esta 'conectada' a otras cuatro placas en sus extremos, lo qeu sucede a un extremo se 'pasa' al otro y as\'i se determina el comportamiento de las fronteras de nuestra placa.

\begin{center}
\includegraphics[width=8cm]{9.pdf}
\end{center}
Igualmente las condiciones iniciales son las mismas.
\includegraphics[width=6cm]{10.pdf}
\includegraphics[width=6cm]{11.pdf}
Gr\'aficas para tiempos intermedio, se puede observar que el comportamiento de las fronteras abiertas como de las peri\'odicas es bastante similar debido a la simetr\'ia de la placa y el cilindro.
\begin{center}
\includegraphics[width=8cm]{12.pdf}
\end{center}
LA gr\'afica final para las condiciones peri\'odicas muestra que el resultado es similar al obtenido en las graficas para fronteras abiertas, como ya se mencion\'o debido a la simetr\'ia del sistema, igualmente se observa que la temperatura aumenta y cuando se llegue al valor para la estabilidad toda la placa se va a encontrar en 100 grados.

\begin{center}
\includegraphics[width=8cm]{Promedio3.pdf}
\end{center}
Al igual que la grafica del promedio de temperaturas para las condiciones abiertas se observa un cambio de temperatura lineal como se esperab debido a la simetria encontrada en el sistema.

\end{document}
