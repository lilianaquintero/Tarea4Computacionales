\documentclass[a4paper,11pt]{article}

\usepackage[latin1]{inputenc}
\usepackage{graphicx}
\usepackage{color}

\begin{document}

\begin{center}
{\Large Tarea 4 m\'etodos computacionales} \\
{\textbf Liliana Quintero -201718286}
\end{center}


\noindent
\section{Gr\'aficas proyectil}

\subsection{Angulo 45 grados}

\begin{center}
\includegraphics[width=8cm]{Plot_ODE_1.pdf} 
Gr\'afica para el proyectil lanzado con un angulo de 45 grados, la distancia m\'axima obtenida es de x=4.26, la altura m\'axima es de y=3.42. Se puede ver como la gravedad y la resistencia del aire forman el tiro parab\'olico y adem\'as como debido a la fricci\'on tanto en x como en y se curva el movimiento de forma que no es una parabola 'perfecta'.
\end{center}

\subsection{Diferentes angulos}

\begin{center}
\includegraphics[width=8cm]{Plot_ODE_2.pdf}
\end{center}
Gr\'afica para el proyectil lanzado con diferentes \'angulos y se puede observar como el \'angulo para el cual se obtuvo la mayor distancia en x fue de 20 grados con el cual la distancia recorrida con el proyectil fue de x=5.23. Como despu\'ues de este \'angulo las distancias recorridas dismiyunes a medida que el \'angulo aumenta, la distancia m\'as corta presentada fue para el \'angulo de 70 grados con el cual se obtuvo una distancia de x=2.22.  Igulamente se observa como el tiro no es perfectamente parab\'olico debido a la fricci\'on presente tanto en el eje x como y.

\section{Gr\'aficas difusi\'on t\'ermica}

Las siguientes gr\'aficas de difusi\'on representan un sistema qeu costa de una piedra caliza que inicialmente se encuentra a temperatura 10 grados y se pone en contacto con una barilla perpendicular en el centro de la misma, y se observa el cambio de la temperatura atravez de la piedra a medida que pasa el tiempo. La piedra tiene un nu espec\'ifico (coeficiente __).


\subsection{Fronteras fijas}

Para las fronteras fijas tenemos que todos los 'bordes' del cuadrado se encuentran siempre a una temperatura de 10 grados.

\begin{center}
\includegraphics[width=8cm]{1.pdf}
\end{center}
Esta es la condici\'on inicial para la piedra, la barilla se encuentra a 100 grados y la piedra a 10 grados para un tiempo t=0.

\includegraphics[width=6cm]{2.pdf}
\includegraphics[width=6cm]{3.pdf}
Estas dos gr\aficas muestran el cambio de la temperatura para dos tiempos intermedios diferentes en los cuales se puede observar el aumento general de temperatura de la misma. En las gr\'aficas se puede observar como se va dando el aumento de la temperatura a lo largo de toda la piedra.

\begin{center}
\includegraphics[width=8cm]{4.pdf}
\end{center}
Esta gr\'afica muestra la condici\'on de equilibrio encontrada, nunca llegan a 100 ya que los extremos son fijos sin embargo llega un punto en el que ya no aumenta o dosminuye m\as la temperatura. Queda en un punto estable.

\subsection{Fronteras abiertas}
Para fronteras abiertas las part\iculas de lso extremos se comportan libremente como el resto de las part\'iculas.

\begin{center}
\includegraphics[width=8cm]{5.pdf}
\end{center}

\includegraphics[width=6cm]{6.pdf}
\includegraphics[width=6cm]{7.pdf}

\begin{center}
\includegraphics[width=8cm]{8.pdf}
\end{center}


\subsection{Fronteras peri\'odicas}
\begin{center}
\includegraphics[width=8cm]{9.pdf}
\end{center}

\includegraphics[width=6cm]{10.pdf}
\includegraphics[width=6cm]{11.pdf}

\begin{center}
\includegraphics[width=8cm]{12.pdf}
\end{center}



\end{document}
