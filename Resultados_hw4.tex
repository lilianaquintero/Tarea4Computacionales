\documentclass[a4paper,11pt]{article}

\usepackage[latin1]{inputenc}
\usepackage{graphicx}
\usepackage{color}

\begin{document}

\begin{center}
{\Large Tarea 4 m\'etodos computacionales} \\
{\textbf Liliana Quintero -201718286}
\end{center}


\noindent
\section{Gr\'aficas proyectil}

\subsection{Angulo 45 grados}

\begin{center}
\includegraphics[width=8cm]{Plot_ODE_1.pdf} 
Gr\'afica para el proyectil lanzado con un angulo de 45 grados, la distancia m\'axima obtenida es de x=4.26, la altura m\'axima es de y=3.42. Se puede ver como la gravedad y la resistencia del aire forman el tiro parab\'olico y adem\'as como debido a la fricci\'on tanto en x como en y se curva el movimiento de forma que no es una parabola 'perfecta'.
\end{center}

\subsection{Diferentes angulos}

\begin{center}
\includegraphics[width=8cm]{Plot_ODE_2.pdf}
\end{center}
Gr\'afica para el proyectil lanzado con diferentes \'angulos y se puede observar como el \'angulo para el cual se obtuvo la mayor distancia en x fue de 20 grados con el cual la distancia recorrida con el proyectil fue de x=5.23. Como despu\'ues de este \'angulo las distancias recorridas dismiyunes a medida que el \'angulo aumenta, la distancia m\'as corta presentada fue para el \'angulo de 70 grados con el cual se obtuvo una distancia de x=2.22.  Igulamente se observa como el tiro no es perfectamente parab\'olico debido a la fricci\'on presente tanto en el eje x como y.

\section{Gr\'aficas difusi\'on t\'ermica}

\subsection{Fronteras fijas}
\begin{center}
\includegraphics[width=8cm]{1.pdf}
\end{center}

\includegraphics[width=6cm]{2.pdf}
\includegraphics[width=6cm]{3.pdf}

\begin{center}
\includegraphics[width=8cm]{4.pdf}
\end{center}


\subsection{Fronteras abiertas}
\begin{center}
\includegraphics[width=8cm]{5.pdf}
\end{center}

\includegraphics[width=6cm]{6.pdf}
\includegraphics[width=6cm]{7.pdf}

\begin{center}
\includegraphics[width=8cm]{8.pdf}
\end{center}


\subsection{Fronteras peri\'odicas}
\begin{center}
\includegraphics[width=8cm]{9.pdf}
\end{center}

\includegraphics[width=6cm]{10.pdf}
\includegraphics[width=6cm]{11.pdf}

\begin{center}
\includegraphics[width=8cm]{12.pdf}
\end{center}



\end{document}
